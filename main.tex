\documentclass{article}

% to-do
% -----
% - Audit for uses of "believe" which is a technical word, related to probability.
% - Be careful with hyphen and en-dash.

% typesetting
\linespread{1.08333} % 10/13 spacing
\frenchspacing\sloppy\sloppypar\raggedbottom

\begin{document}

\noindent
\textbf{Bets among friends:\\ Algorithms to enable mutually acceptable betting}

\medskip
\noindent
\textbf{David W. Hogg}\\
\textit{Center for Cosmology and Particle Physics, Department of Physics, New York University}

\paragraph{Abstract:}
Friends sometimes like to bet on things.
Here we address the question of how to structure such bets such that they are both fair and mutually beneficial.
We consider two kinds of bets between two friends.
In one, there is a well-defined proposition, and the question is:
How to set the odds?
In another, there is an even-odds bet that some number will be over or under some value, and the question is:
How to set the over--under value?
We propose solutions that involve the two players submitting secret but binding bids to an escrow service.

\paragraph{Background information:}
I am a bettor, and I place bets at a book.
The book offers me bets and odds on those bets.
For example, the book might offer odds\footnote{%
Different books present to the customer odds in different formats.
The odds that I write as 1.8:1 will be written in ``American format'' as +1.80.
It will be written in ``European format'' as 2.800.
And it will be written at the horse track as 1.80.
Similarly, the odds that I write as 0.5:1 or 1:2.0 will be written in American format as -2.00 (huh?),
European format as 1.500,
and at the track as 0.5.
None of this notation matters to us right now, but I'm putting it here to remind people that there are
different languages in play.}
of 1.8:1 that the Argos will win their next game.
That means that if I bet 100\,USD to win and the Argos win, I will be paid back my 100\,USD plus 180\,USD in winnings for a total of 280\,USD.
If I bet 100\,USD to win and the Argos lose, I will lose my 100\,USD.

When contemplating gambling, it is useful (and, apparently, optimal\footnote{HOGG CITE.}) to reason like a Bayesian.
From a Bayesian perspective, probabilities are \emph{degrees of belief}.
If I say that there is a 35-percent chance (say) that the Argos will win their next game, what I mean is:
I don't know whether the Argos will win, and I don't think it is particularly likely,
but I am willing to bet that they will win only if you give me sufficient odds.
For Bayesians, probabilities don't translate into frequencies of events, they translate into betting
behaviors.

Let's make this quantitative.
The (only) way to understand a Bayesian probability\footnote{%
Sometimes people like to think of probabilities as frequencies, but there is no good frequentist description
of ``the probability that the Argos will win their next game.''
After all, in what universe do we get to see this singular event over and over again (to develop or observe frequencies)?
For this reason, I think this betting-odds way of thinking about Bayesian probabilities is the only consistent,
quantitative statement that can be made about one's probabilities or degrees of belief.}
it to ask at what odds would the Bayesian place a bet.
So, for instance, if I say there is a 35-percent chance that the Argos will win their next game,
then I should be willing to bet (a small amount of money) that they will win if the book offers me odds
of $R:1$ to win with $R>1.86$
and I should be willing to bet that they will lose if the book offers
me odds of $R:1$ to lose with $R>0.54$.
There is a direct translation between odds and probabilities, such that if I say that the probability
that some proposition $H$ will happen (or is true) is $P_H$, then I should be willing to bet that $H$
will happen if I am offered odds $R:1$ with $R>P_H/(1 - P_H)$.
I should be willing to bet that $H$ won't happen (or is not true) if I am offered odds $R:1$ with $R>(1-P_H)/P_H$.

One of the interesting facts about these odds is that, if I am a rational Bayesian,
and if the stakes are not very large\footnote{%
It is hard to know whether I would bet a decade of salary on something, even if I thought the odds were in my favor.
So in what follows we only consider small bets, which are small in the sense of ``not meaningfully changing my standard of living.''}
then the threshold $P/(1-P)$ on the odds $R$ at which I am willing to bet in favor of the proposition is
precisely the inverse of the threshold $(1-P)/P$ on the odds $R$ at which I am willing to bet against the proposition.
That's going to be useful, below.

\paragraph{Prop bet setup:}
We have two friends, $A$ and $B$, who want to bet on an outcome $H$ of some event.
They are friends, so they want to structure their bet so both friends consider the betting conditions to be fair.
There is some number $R_H^{(A)}$ such that, if offered odds $R:1$ with $R>R_H^{(A)}$, friend $A$ would bet on $H$.
There is some other number $R_{\not H}^{(A)}$ such that, if offered odds $R:1$ with $R>R_{\not H}^{(A)}$, friend $A$
would bet on the opposite of $H$, or not-$H$ or $\not H$.
If $A$ is rational, if $A$ is down to bet, and if the stakes are not large, then it should be the case that
$R_{\not H}^{(A)} \approx 1/R_H^{(A)}$.

Similarly for friend $B$, there should be a threshold $R_H^{(B)}$ such that, if offered odds $R:1$ with $R>R_H^{(B)}$, friend $B$ would bet on $H$.
There should be a threshold $R_{\not H}^{(B)}$ such that, if offered odds $R:1$ with $R>R_{\not H}^{(B)}$, friend $B$
would bet on $\not H$.
And it should be that $R_{\not H}^{(B)} \approx 1/R_H^{(B)}$.

Now imagine that both friends, $A$ and $B$ decide on a proposition $H$ and a ``stakes'' $q$.
The proposition $H$ needs to be a well-defined yes--no question that will get unambiguously decided in the near future.\footnote{%
I say ``near future'' here because we don't want to get into discount-rate or interest-rate considerations,
which might differentiate $A$ and $B$ in weird ways.}
The stakes $q$ is the scale of the money involved, denominated in money units.
In detail we are going to set up our bet such that the total amount of money at stake is $2\,q$.
That is, the sum of what friend $A$ bets and what friend $B$ bets will be exactly $2\,q$.
This is a choice made here, and not in any sense a general definition of the word ``stakes.''

\paragraph{Prop algorithm:}

\paragraph{O/U bet setup:}
There is another kind of bet that friends $A$ and $B$ might want to make on an event.
There might be a \emph{number} $y$ generated by the event, and they want to bet on the \emph{value} of that number $y$.
One nice structure is to choose a threshold $x$ for that number, such that a bet that $y>x$ is even odds or $1:1$, and
a bet that $y<x$ is also even odds.

\paragraph{O/U algorithm:}
\end{document}
